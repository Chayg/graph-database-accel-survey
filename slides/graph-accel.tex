%%%%%%%%%%%%%%%%%%%%%%%%%%%%%%%%%%%%%%%%%
% Beamer Presentation
% LaTeX Template
% Version 1.0 (10/11/12)
%
% This template has been downloaded from:
% http://www.LaTeXTemplates.com
%
% License:
% CC BY-NC-SA 3.0 (http://creativecommons.org/licenses/by-nc-sa/3.0/)
%
%%%%%%%%%%%%%%%%%%%%%%%%%%%%%%%%%%%%%%%%%

%----------------------------------------------------------------------------------------
%	PACKAGES AND THEMES
%----------------------------------------------------------------------------------------

\documentclass{beamer}

\mode<presentation> {

% The Beamer class comes with a number of default slide themes
% which change the colors and layouts of slides. Below this is a list
% of all the themes, uncomment each in turn to see what they look like.

%\usetheme{default}
%\usetheme{AnnArbor}
%\usetheme{Antibes}
%\usetheme{Bergen}
\usetheme{Berkeley}
%\usetheme{Berlin}
%\usetheme{Boadilla}
%\usetheme{CambridgeUS}
%\usetheme{Copenhagen}
%\usetheme{Darmstadt}
%\usetheme{Dresden}
%\usetheme{Frankfurt}
%\usetheme{Goettingen}
%\usetheme{Hannover}
%\usetheme{Ilmenau}
%\usetheme{JuanLesPins}
%\usetheme{Luebeck}
%\usetheme{Madrid}
%\usetheme{Malmoe}
%\usetheme{Marburg}
%\usetheme{Montpellier}
%\usetheme{PaloAlto}
%\usetheme{Pittsburgh}
%\usetheme{Rochester}
%\usetheme{Singapore}
%\usetheme{Szeged}
%\usetheme{Warsaw}

% As well as themes, the Beamer class has a number of color themes
% for any slide theme. Uncomment each of these in turn to see how it
% changes the colors of your current slide theme.

%\usecolortheme{albatross}
%\usecolortheme{beaver}
%\usecolortheme{beetle}
%\usecolortheme{crane}
%\usecolortheme{dolphin}
%\usecolortheme{dove}
%\usecolortheme{fly}
%\usecolortheme{lily}
%\usecolortheme{orchid}
%\usecolortheme{rose}
%\usecolortheme{seagull}
%\usecolortheme{seahorse}
%\usecolortheme{whale}
%\usecolortheme{wolverine}

%\setbeamertemplate{footline} % To remove the footer line in all slides uncomment this line
%\setbeamertemplate{footline}[page number] % To replace the footer line in all slides with a simple slide count uncomment this line

%\setbeamertemplate{navigation symbols}{} % To remove the navigation symbols from the bottom of all slides uncomment this line
}
\usepackage{setspace}
\usepackage{graphicx} % Allows including images
\usepackage{booktabs} % Allows the use of \toprule, \midrule and \bottomrule in tables
\setbeamertemplate{caption}[numbered]
%----------------------------------------------------------------------------------------
%	TITLE PAGE
%----------------------------------------------------------------------------------------

\title[]{Graph Acceleator Review} 
\author[]{
    Cheng Liu 
}
\institute {
\medskip
}
\date{\today} % Date, can be changed to a custom date

\graphicspath{{./figures/}} 
\begin{document}

\begin{frame}
\titlepage % Print the title page as the first slide
\end{frame}

%----------------------------------------------------------------------------------------
%	PRESENTATION SLIDES
%----------------------------------------------------------------------------------------

%------------------------------------------------
\section{Vertex-Centric Parallel Graph Processing} 
%------------------------------------------------
\begin{frame}[t]
\frametitle{Basic Vertex Processing Framework}
    Here is the basic vertex-centric processing flow.
    \begin{itemize}
        \item Initialize Active Vertex List
        \item For each active vertex, get associated Edge List
        \item For each edge in the Edge List, update associated vertices
        \item Update Active Vertex List
    \end{itemize}
    Iteratively perform the vertex processing with BSP model until the computing converges. 
    Graph operation abstraction in different graph frameworks: Advance-Compute-filter,
    Gather-Scatter-Apply, EdgeMap-VertexMap, ProcessEdge-Reduce-Apply ...
\end{frame}

%------------------------------------------------
\section{Typical Parallelization Methods}
%------------------------------------------------
\begin{frame}[t]
    \frametitle{General Graph Optimization Rules}
    \begin{itemize}
        \item Exploit locality of the graph data
        \item Reduce amount of data tranfer or computing
        \item Explore more parallelism 
        \item Explore load balancing strategies
    \end{itemize}
\end{frame}

\begin{frame}[t]
    \frametitle{Graph Partition}
    The graph are divided into sub-graphs based on various strategies.
    \begin{itemize}
        \item \textbf{Graph slicing:} Graph IDs are used in
            \cite{hamgra2016graphicionado} to divide the vertices and edges into
            different groups. Dependent vertices are replicated. 
        \item \textbf{Interval and Shard:} Graph IDs are reordered and equally
            divided as intervals. Based on the vertex intervals, the edges are
            further split into shards. Basically the graph are divided in two
            dimensions \cite{dai2016fpgp, Chi2015NXgraph}. Note that it needs
            pre-processing. 
    \end{itemize}
\end{frame}

\begin{frame}[t]
    \frametitle{Vertex Related Optimization Techniques}
    The optimization techniques concentrates on the vertices in the active
    vertex list. The basic idea is to divide the vertices in the list into
    groups with different priority and process them separately. This helps on
    both balance the load on parallel computing architectures and reduce the
    computing work in some of the applications. 
    \begin{itemize}
        \item Bottom-up (push) and Top-down (pull) Vertex Traversal \cite{wang2016gunrock, Shun2013ligra} 
        \item Remove Redundant Vertices in Vertex Active List \cite{wang2016gunrock}
        \item Priority Queue for Active Vertex List \cite{Shun2013ligra}
        \item Near-Far pile, delta bucketing \cite{wang2016gunrock, meyer2003delta}  
    \end{itemize}
\end{frame}

\begin{frame}[t]
    \frametitle{Edge Related Optimization Techniques}
    The edge optimization techniques mostly aim to achieve load balancing on the
    parallel architectures. 
    \begin{itemize}
        \item Naive vertex based allocation
        \item Group Blocking
        \item CTA (Cooperative Thread Array) + WARP \cite{wang2016gunrock, merrill2012scalable}
        \item Equally edge partition \cite{wang2016gunrock}
    \end{itemize}
\end{frame}

\begin{frame}[t]
    \frametitle{Memory Access Optimization}
    \begin{itemize}
        \item \textbf{Double-Buffer:} Iteratively swap vertex buffers of
            vertex data in current iteration and next iteration.
        \item \textbf{eDRAM:} For random edge or vertex access, a large eDRAM
            can be used as a on-chip scratchpad memory \cite{hamgra2016graphicionado}. 
        \item \textbf{Pre-Fetch:} When all the edges or vertices are to be
            accessed, sequentially accessing all the data with pre-fetching can
            hide memory access latency completely \cite{hamgra2016graphicionado}.
    \end{itemize}
\end{frame}

\section{Reference}
\begin{frame}[t]
    \frametitle{Reference}
    \tiny
    \bibliographystyle{abbrv}
    \bibliography{refs}
\end{frame}
\end{document} 
