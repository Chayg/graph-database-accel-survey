\section{Gunrock}
Gunrock \cite{wang2016gunrock} is a data-centric graph processing framework which mainly handles
graphs that can be expressed with iterative convergent processes. It is also
based on the BSP model. Basically, the graph
problems are divided into steps and each step must be synchronized.
Advance-filter-compute are the typical primitive steps used in Gunrock. 
These operations are performed on top of graph frontiers which is used in many
graph processing framework. Major optimization strategies used in Gunrock are
listed below.

\begin{itemize}
    \item Kernel fusion: Inspired by the primitive GPU optimization strategy
        which leverages the producer-consumer locality between operations,
        Gunrock tries to integrate advance, filter and user-specific functor
        into a thread to improve performance as well as memory access
        efficiency.
    \item Workload balance: Gunrock's advance step which is applied on graph
        frontier result in severe load balancing problem especially for graphs
        with power-law distribution just as any other similar framework. To
        solve this problem, Gunrock roughly divides the vertices in the
        frontier into three groups based on the size of the associated neighbor
        lists. Vertices with larger size of neighbor lists will be executed in
        parallel in a cooperative thread array (CTA). Vertices with medium size
        will be executed in parallel in a warp. Vertices with small size will be
        executed in separate threads. When multiple vertices are assigned to a
        CTA or warp, the vertex with largest size of associated neighbor list
        will be distributed to all the threads of the CTA or warp and be executed
        first. Then all the vertices will be executed sequentially with the same
        logic.
    \item Idempotent vs. non-idempotent operations: As vertices in current
        frontier may share the same neighbors, there are duplicate vertices when
        producing the next frontier. Gunrock removes some of the duplicate
        vertices with inexpensive heuristics for applications that allows
        duplicated vertices in frontier. This strategy helps to improve
        performance. For applications that can't tolerate duplicated vertices,
        it can also remove duplicate vertices completely. 

    \item Pull and push traversal: This is the same with the top-down
        and bottom-up traversal strategy used in Ligra.

    \item Priority queue: Usually all the vertices in the frontier are treated
        equally in BSP model while Gunrock divides them into two queues
        based on a criterion to save work.  

\end{itemize}

\section{Work-efficient Parallel GPU Methods for SSSP}
The algorithm developed in this work \cite{Davidson2014work-efficient} is based
on delta-stepping algorithm. It consists of the following steps.
\begin{itemize}
    \item The vertices are divided into one or multiple buckets. Vertices within
        a bucket is processed in parallel.
    \item Traverse the vertices in a bucket. Load balancing is the key challenge
        in this step.
    \item Decide vertices to be processed in next iteration.
\end{itemize}

This work comes from the same group of Gunrock and it focuses on SSSP
optimization on GPUs. Although the optimization techniques used target
SSSP, they are generalized and ported to Gunrock. Thus this work is reviewed
for more details about the graph optimization techniques. Here are the highlights of
this SSSP optimization techniques.

\begin{itemize}
    \item load balanced graph traversal:
        \subitem Group blocking: The edges of the vertices within a block are
        stripped from each vertex's edge list and processed by a cooperative
        thread array (CTA) in parallel. Threads within a block is load-balanced,
        but load-imbalance may still exist between the blocks. Particularly,
        when the vertex degree is small, this method is not efficient.

        \subitem CTA+Warp+Scan \cite{merrill2012scalable}: The basic idea is to divide the vertices into
        three categories based on the size of the edge list. The method is
        applied in Gunrock as described in last section. 

        \subitem Edge partition: Instead of grouping equal number of vertices in
        each block, this method organizes the groups of edges with equal length
        to ensure strict load balance within a block. 
    \item Work organization which essentially decides the vertices to be
        processed in next iteration. Again three different methods are proposed.
        \subitem Workfront sweep: The basic idea of Workfront is to compute on
        vertex frontier instead of all the vertice in Bellman-Ford. In addition,
        with the frontier queue and a vertex id table, redundant vertice in the
        queue are completely removed to minmize the computing work.

        \subitem Near-Far Pile: This is used in Ligra and also known as priority
        queue strategy in Gunrock. The basic idea is to process some of the
        vertices in the work queue or frontier first which helps to save the
        work of the computing. Particularly, the Near-Far method divides the
        vertices in the work queue in two piles based on the distances to the
        source. The distance threshold is named as delta which can be
        customized. When the vertices in the near pile are processed, vertices
        in the far pile will be analyzed to update the new near pile with the
        threshold increased by anther delta. Meanwhile, duplicated vertices in
        the far piles will be removed. 

        \subitem Bucketing with far pile: This method follows the same design
        philosophy with Near-Far pile. While delta in near-far method may impose
        diverse number of vertices in near pile in different iterations. The
        bucketing method selects a determined number of vertices for the near
        pile when the number of vertices that meets the distance criteria is
        than the predefined number (i.e. 1\/10 of the active vertices in this
        work). The rest active vertices are considered as the far pile. In next
        iteration, the vertices in the far pile are further selected with a
        increased distance criteria similarly. This method will ensure the
        number of near pile varies in a relatively small range which fits 
        the GPU hardware. 
\end{itemize}
By adding priority information to the vertices in the active list, we can avoid
updating or computing vertices associated with current vertices with long edges,
as they will probably be updated by other paths with smaller edges and more
hops. Essentially, this reduces the amount of computing and the overall run-time
can be reduced as long as the GPU processing elements remain high utilization.

Here is a summary of relevant SSSP algorithms in reference papers. Some of them
are the basis of the algorithms proposed in this work.
\begin{table}[!hbp]
    \begin{center}
    \begin{tabular}{cccc}
        \toprule[2pt]
         \textbf{Algorithms} & \textbf{Wk. Complexity} & \textbf{Type} &
         \textbf{Parallelism}\\
        \hline
        Dijkstra & $O(v\log{v}+e)$ & General & Serial \\
        Bellman-Ford & $O(ve)$ & High Degree & Parallel \\
        Delta Step & $O(v\log{v}+e)$ & General & Coarse Parallel \\
        PHAST & $O(v\log{v}+e)$ & Low Degree & Preprocessing Parallel \\
        \bottomrule[2pt]
    \end{tabular}
    \end{center}
    \caption{SSSP Algorithms}
\end{table} 

Dijkstra algorithm implemented with a priority queue is efficient as a
sequential algorithm but expose little parallelism for parallel computing
architectures.

PHAST \cite{delling2013phast}: It has a Dijkstra-like preprocessing step to pre-compute distances to
vertices of high-degree. Then the Dijkstra algorithm can start from these highly ranked
vertices in parallel. This algorithm works well on low-degree and high-diameter
graphs. (More details will be added later.)

Delta Step \cite{meyer2003delta}: Instead of processing one vertex at a time in
Dijkstra algorithm, it groups vertices in buckets and process vertices in a
bucket in parallel. In delta-stepping, the vertices are grouped into buckets
depending on distances of the vertices from the source.

Major challenges for Delta step algorithm on GPUs.
\begin{itemize}
    \item Delta-stepping's bucket implementation requires dynamic array that can
        be quickly resized in parallel.
    \item Fine-grained renaming and moving vertices between buckets are
        difficult and inefficient.
    \item GPU memory hierarchy is not well explored.
\end{itemize}

Bellman-Ford: It is a standard parallel algorithm for SSSP problem. Each vertex
maintains the distance to the source and has neighbor vertices information
updated iteratively. The algorithm completes when the algorithm converges. 
This algorithm suffers load imbalance for graphs with power-law distribution.
Race condition occurs when the update is parallelized and atomic update is
needed.

\subsection{FPGA}

\subsection{NXgraph}

